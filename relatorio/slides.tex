\documentclass{beamer}

% --- PACOTES ---
\usepackage[utf8]{inputenc}
\usepackage[brazil]{babel}
\usepackage{graphicx}

% --- TEMA E CORES ---
% Você pode experimentar outros temas, como: Warsaw, Boadilla, Madrid, etc.
\usetheme{Madrid}
\usecolortheme{beaver}

% --- INFORMAÇÕES DO TÍTULO ---
\title[Panoramas]{MO446 - Visão Computacional \\ Projeto de Panoramas}
\author{Luciana P. Nedel \and Rafael H. Bordini \and Flávio R. Wagner \and Jomi F. Hübner}
\institute[UFRGS, Durham, FURB]{}
\date{\today}

\begin{document}

% --- SLIDE DE TÍTULO ---
\begin{frame}
  \titlepage
\end{frame}

% --- SLIDE DE ROTEIRO (SUMÁRIO) ---
\begin{frame}
  \frametitle{Roteiro da Apresentação}
  \tableofcontents
\end{frame}

% ===================================================================
\section{Coleta das Imagens}
% ===================================================================
\begin{frame}
  \frametitle{Etapa 1: Coleta das Imagens}
  \begin{itemize}
    \item<1-> Capturar um conjunto de imagens com sobreposição parcial (ex: 30-50%).
    \item<2-> Manter o centro óptico da câmera o mais fixo possível.
    \item<3-> Garantir boas condições de iluminação e foco.
    \item<4-> Documentar o dispositivo e o local da captura.
  \end{itemize}
\end{frame}

% ===================================================================
\section{Detecção e Extração de Características}
% ===================================================================
\begin{frame}
  \frametitle{Etapa 2: Detecção e Extração de Características}
  \begin{itemize}
    \item<1-> O objetivo é encontrar pontos de interesse (keypoints) em cada imagem.
    \item<2-> Usamos algoritmos como SIFT, ORB ou AKAZE para detectar e descrever esses pontos.
    \item<3-> \textbf{Exemplo:} Visualização dos keypoints detectados sobre uma imagem.
  \end{itemize}
  \begin{figure}
    % Substitua 'placeholder.jpg' pelo nome do seu arquivo de imagem
    % \includegraphics[width=0.7\textwidth]{placeholder.jpg}
  \end{figure}
\end{frame}

% ===================================================================
\section{Emparelhamento de Características}
% ===================================================================
\begin{frame}
  \frametitle{Etapa 3: Emparelhamento de Características}
  \begin{itemize}
    \item<1-> Encontrar correspondências entre os keypoints de imagens adjacentes.
    \item<2-> Utilizar um algoritmo de casamento, como \textit{Brute-Force Matcher}.
    \item<3-> Aplicar filtros para remover casamentos incorretos, como o \textit{Ratio Test} de Lowe.
  \end{itemize}
\end{frame}

% ===================================================================
\section{Estimação de Homografia e Alinhamento}
% ===================================================================
\begin{frame}
  \frametitle{Etapa 4: Estimação de Homografia e Alinhamento}
  \begin{itemize}
    \item<1-> A \textbf{homografia} é uma matriz 3x3 que mapeia os pontos de um plano de imagem para outro.
    \item<2-> Usamos o algoritmo \textbf{RANSAC} para estimar uma homografia robusta, mesmo com \textit{outliers}.
    \item<3-> Com a homografia, alinhamos (warp) uma imagem à perspectiva da outra.
  \end{itemize}
\end{frame}

% ===================================================================
\section{Composição e Blending}
% ===================================================================
\begin{frame}
  \frametitle{Etapa 5: Composição e Blending}
  \begin{itemize}
    \item<1-> \textbf{Composição:} Unir todas as imagens alinhadas em um único "canvas".
    \item<2-> \textbf{Blending:} Aplicar técnicas de suavização (ex: \textit{feathering}, \textit{multiband blending}) para tornar as emendas invisíveis.
    \item<3-> O resultado é o panorama final!
  \end{itemize}
\end{frame}



\end{document}
